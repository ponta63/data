%#!platex
\documentclass[
  platex, dvipdfmx,  % ワークフローは必ず明示的に指定する
]{nlp2021}
%#!uplatex
%\documentclass[uplatex,dvipdfmx]{nlp2021}
%#!lualatex
%\documentclass[lualatex]{nlp2021}


% パッケージ
\usepackage{xcolor}  %
\usepackage{graphicx}  % グラフィックス関連
\usepackage{pxrubrica}        % ルビ
\usepackage{url}
\usepackage[square,numbers]{natbib} % 参考文献のフォントサイズを変更する用


%% option 不要な場合はコメントアウト
\usepackage{bxjalipsum}       % ダミーテキスト
\usepackage{hyperref}
\usepackage{pxjahyper}
\hypersetup{
    colorlinks=true, 
    citecolor=blue, 
    linkcolor=blue,
    pdfborder={0 0 0},
}
\def\bibfont{\small} % 参考文献のフォントサイズを指定


% 著者用マクロをここに入れる
\newcommand{\pkg}[1]{\textsf{#1}}
\newcommand{\code}[1]{\texttt{#1}}
\newcommand{\comment}[1]{\textcolor{red}{#1}}
%%%%%%%

\title{NLP2021-好きなタイトル}
\author{
AAA \quad\quad BBB \quad\quad CCC\\
XXX株式会社 \\
\texttt{ \{aaa, bbb, ccc\}@linecorp.com }
}
  
  
\begin{document}

\maketitle

\section{はじめに}
あああああああああああああああああああ
あああああああああああああああああああ
あああああああああああああああああああ
あああああああああああああああああああ
あああああああああああああああああああ
あああああああああああああああああああ
あああああああああああああああああああ

あああああああああああああああああああ
あああああああああああああああああああ
あああああああああああああああああああ
あああああああああああああああああああ
あああああああああああああああああああ
あああああああああああああああああああ
あああああああああああああああああああ

あああああああああああああああああああ
あああああああああああああああああああ
あああああああああああああああああああ
あああああああああああああああああああ
あああああああああああああああああああ
あああああああああああああああああああ
あああああああああああああああああああ

あああああああああああああああああああ
あああああああああああああああああああ
あああああああああああああああああああ
あああああああああああああああああああ
あああああああああああああああああああ
あああああああああああああああああああ
あああああああああああああああああああ

あああああああああああああああああああ
あああああああああああああああああああ
あああああああああああああああああああ
あああああああああああああああああああ
あああああああああああああああああああ
あああああああああああああああああああ
あああああああああああああああああああ

あああああああああああああああああああ
あああああああああああああああああああ
あああああああああああああああああああ
あああああああああああああああああああ
あああああああああああああああああああ


\section{関連研究}
あああああああああああああああああああ
あああああああああああああああああああ
あああああああああああああああああああ
あああああああああああああああああああ
あああああああああああああああああああ
あああああああああああああああああああ
あああああああああああああああああああ

あああああああああああああああああああ
あああああああああああああああああああ
あああああああああああああああああああ
あああああああああああああああああああ
あああああああああああああああああああ
あああああああああああああああああああ
あああああああああああああああああああ

あああああああああああああああああああ
あああああああああああああああああああ
あああああああああああああああああああ
あああああああああああああああああああ
あああああああああああああああああああ
あああああああああああああああああああ
あああああああああああああああああああ

あああああああああああああああああああ
あああああああああああああああああああ
あああああああああああああああああああ
あああああああああああああああああああ
あああああああああああああああああああ
あああああああああああああああああああ
あああああああああああああああああああ

%%%%%%%%%%%%%%%%%%%%%%%%%%%%
%%%%%%%%%%%%%%%%%%%%%%%%%%%%
%%%%%%%%%%%%%%%%%%%%%%%%%%%%
\section{アプローチ}
あああああああああああああああああああ
あああああああああああああああああああ
あああああああああああああああああああ
あああああああああああああああああああ
あああああああああああああああああああ
あああああああああああああああああああ
あああああああああああああああああああ

あああああああああああああああああああ
あああああああああああああああああああ
あああああああああああああああああああ
あああああああああああああああああああ
あああああああああああああああああああ
あああああああああああああああああああ
あああああああああああああああああああ



%%%%%%%%%%%%%%%%%%%%%%%%%%%%
%%%%%%%%%%%%%%%%%%%%%%%%%%%%
%%%%%%%%%%%%%%%%%%%%%%%%%%%%
\section{実験}

\subsection{データセット}
あああああああああああああああああああ
あああああああああああああああああああ
あああああああああああああああああああ
あああああああああああああああああああ
あああああああああああああああああああ
あああああああああああああああああああ
あああああああああああああああああああ
\subsection{評価指標}
あああああああああああああああああああ
あああああああああああああああああああ
あああああああああああああああああああ
あああああああああああああああああああ
あああああああああああああああああああ
あああああああああああああああああああ
あああああああああああああああああああ

\subsection{実験設定}
あああああああああああああああああああ
あああああああああああああああああああ
あああああああああああああああああああ
あああああああああああああああああああ
あああああああああああああああああああ
あああああああああああああああああああ
あああああああああああああああああああ

\section{実験結果}
あああああああああああああああああああ
あああああああああああああああああああ
あああああああああああああああああああ
あああああああああああああああああああ
あああああああああああああああああああ
あああああああああああああああああああ
あああああああああああああああああああ

あああああああああああああああああああ
あああああああああああああああああああ
あああああああああああああああああああ
あああああああああああああああああああ
あああああああああああああああああああ
あああああああああああああああああああ
あああああああああああああああああああ

あああああああああああああああああああ
あああああああああああああああああああ
あああああああああああああああああああ
あああああああああああああああああああ
あああああああああああああああああああ
あああああああああああああああああああ
あああああああああああああああああああ

あああああああああああああああああああ
あああああああああああああああああああ
あああああああああああああああああああ
あああああああああああああああああああ
あああああああああああああああああああ
あああああああああああああああああああ
あああああああああああああああああああ

\section{おわりに}
あああああああああああああああああああ
あああああああああああああああああああ
あああああああああああああああああああ
あああああああああああああああああああ
あああああああああああああああああああ
あああああああああああああああああああ
あああああああああああああああああああ

あああああああああああああああああああ
あああああああああああああああああああ
あああああああああああああああああああ
あああああああああああああああああああ
あああああああああああああああああああ
あああああああああああああああああああ
あああああああああああああああああああ

あああああああああああああああああああ
あああああああああああああああああああ
あああああああああああああああああああ
あああああああああああああああああああ
あああああああああああああああああああ
あああああああああああああああああああ
あああああああああああああああああああ

あああああああああああああああああああ
あああああああああああああああああああ
あああああああああああああああああああ
あああああああああああああああああああ
あああああああああああああああああああ
あああああああああああああああああああ
あああああああああああああああああああ




%%%%%%%%%%%%%%%%%%%%%%%%%%%%%%%%%%%%
%%%%%%%%%%%%%%%%%%%%%%%%%%%%%%%%%%%%
%%%%  ここまでが本文 4ページ以内
% 参考文献
\bibliographystyle{junsrt}
\bibliography{reference}


%%%%%%%%%%%%%%%%%%%%%%%%%%%%%%%%%%%%
%%%%%%%%%%%%%%%%%%%%%%%%%%%%%%%%%%%%
%%%%  ここまでが本文+参考文献 5ページ以内

% 付録(Appendix)
% 付録を付けない場合は、以下\end{document}以外を全てをコメントアウトする.
% 本文、参考文献に続けて作成する場合は、必ず \clearpage して新たなページとする
\clearpage
% 付録は別ツールで作成して、後で本文PDFに追加する方式でもよい
\appendix
% ここ以降はフォーマットを自由に変更可能
\onecolumn % onecolumnにしたい時の例

%文字フォント等の制限無し.余白サイズのみ規定あり.
\section{Appendix}
あああああああああああああああああああ
あああああああああああああああああああ
あああああああああああああああああああ
あああああああああああああああああああ
あああああああああああああああああああ
あああああああああああああああああああ
あああああああああああああああああああ

あああああああああああああああああああ
あああああああああああああああああああ
あああああああああああああああああああ
あああああああああああああああああああ
あああああああああああああああああああ
あああああああああああああああああああ
あああああああああああああああああああ

あああああああああああああああああああ
あああああああああああああああああああ
あああああああああああああああああああ
あああああああああああああああああああ
あああああああああああああああああああ
あああああああああああああああああああ
あああああああああああああああああああ

あああああああああああああああああああ
あああああああああああああああああああ
あああああああああああああああああああ
あああああああああああああああああああ
あああああああああああああああああああ
あああああああああああああああああああ
あああああああああああああああああああ



\end{document}
